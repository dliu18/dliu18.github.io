% LaTeX file for resume
% This file uses the resume document class (res.cls)
\documentclass[margin]{res} 

\usepackage{hyperref}
% the margin option causes section titles to appear to the left of body text 
\textwidth=5.0in % increase textwidth to get smaller right margin
%\usepackage{helvetica} % uses helvetica postscript font (download helvetica.sty)
%\usepackage{newcent}   % uses new century schoolbook postscript font 

\begin{document} 

\name{David Mingfei Liu\\ \hline} % the \\[12pt] adds a blank line after name 

\begin{resume} 
\hspace*{-\hoffset} 3284 Frist Campus Center \hfill dml3@princeton.edu \\
\hspace*{-\hoffset} Princeton, NJ 08544 \hfill (571)-528-3466 \\ 

%\section{Objective} Research internship in the areas of mobile computing/networking. 
 
%\section{Objective} 
%Graduate studies (PhD) at the University of Illinois at Urbana-Champaign in the field of Computer Science in the area of networked systems.

\begin{itemize}
\section{Education} 

\item
\textbf{Princeton University\\}
\emph{B.S.E in Computer Science + Certificate in Statistics and Machine Learning (2018)} \\
Major GPA \textbf{3.84/4.0}, Overall GPA \textbf{3.70/4.0}\\
Coursework includes: Probability and Stochastic Systems, Machine Learning and AI, Neural Networks, Analysis of Big Data, Social Networks, Fairness in Machine Learning, Applied Social Statistics, Applications in Econometrics.
\item
\textbf{Thomas Jefferson High School for Science and Technology} (2014)
\end{itemize}

\begin{itemize} \itemsep -1pt  % reduce space between items
\section{Projects}
\item 
{\bf Replicating and ensembling models from the fragile families challenge} \\
\emph{Advisor: Matthew Salganik} \emph{Aug 2017  - Present }\\
Constructing a community model of both social science and machine learning approaches to predicting the outcomes of disadvantaged children. Peer-reviewed paper on computational reproducibility in progress. \\
\url{https://github.com/dliu18/papers/blob/master/fragile_families.pdf} \\

\item
{\bf Analyzing language trends in post-secondary FERPA notices} \\
\emph{Advisor: Arvind Narayanan} \emph{Feb 2017 - Jun 2017}\\
Scraped education policy documents from the web and used SVM classification to isolate FERPA notices. Data used to survey the current state of education policy compliance.\\
\url{https://github.com/dliu18/papers/blob/master/legal_nlp.pdf}\\

\item
{\bf Sentiment analysis of archived collegiate newspaper headlines}\\
\emph{Advisor: Brian Kernighan} \emph{Sep 2016 - Jan 2017}\\
Created an interactive, public n-gram viewer based on the electronic archives of Princeton's student-run campus newspaper. \\
\url{https://github.com/dliu18/papers/blob/master/digital_humanities.pdf}
\end{itemize}

\begin{itemize}
\section{Industry \\Internships}
\item
{\bf The New York Times - {\em{Software Intern}}  \hfill  Summer 2017}\\
Architected backend supporting the renovation of the “My Account” page of the New York Times, increasing site usability. Renovated site launched in October 2017. Initiated side project with the NYT Data Science team, improving visualization of readership data from embedded static files to native, interactive Google Charts.

\item
{\bf HookLogic Inc. - {\em{Data Science Intern}}  \hfill Summer 2016}\\
Reduced data analysis overhead time for team by initiating Hive queries on AWS clusters to analyze performance of machine learning models.
\end{itemize}


\begin{itemize}
\section{Awards  and Honors}
\item
{\bf {Dean's List:}} Placed on college list on the basis of cumulative academic standing following junior year.  {\bf \hfill Jun 2017}

\item
{\bf {Daily Princetonian Award for Excellence in Sports Writing:}} Annual award given to staff member demonstrating superior writing. {\bf \hfill Jun 2016}

\item
{\bf {Residential College Advisor Scholarship:}} One of 30 campus advisers chosen from class on the basis of leadership and character. Responsibilities include building community among 20 freshmen for one year. Monetary compensation of \$20k, covering room and board. {\bf \hfill Jun 2016}

\end{itemize}

\begin{itemize}
\section {Co-curricular Activities}
\item
{\bf {Peer Academic Advisor, Princeton University:}} Provide academic assistance to freshmen and sophomores with choosing courses, majors and independent projects at Princeton University {\bf \hfill 2016 - 2017}
\item
{\bf {Undergraduate Lab Teaching Assistant, Princeton University:}} Provide debugging and conceptual assistance on programming assignments for introductory Computer Science courses; assist the head Lab TA in the smooth functioning of the Lab TA program {\bf \hfill 2015 - 2016}
\end{itemize}

\begin{itemize}
\section {Extracurricular Activities}
\item
{\bf {Sports Editor, \textit{The Daily Princetonian}:}} Led section of a dozen staff writers to publish a daily paper covering Princeton athletics news and issues of concern. Articles can be found at \url{http://www.dailyprincetonian.com/staff/david-liu}, featuring sports data analytics and visualization.  {\bf \hfill 2014 - 2017}

\item
{\bf {Co-President, Princeton Brown Food Co-operative:}} Point of contact for self-sustaining student body, in which members cook meals for the co-operative on a weekly basis.  {\bf \hfill 2016 - Present}
% Brown Co-op 
% Daily Prince
% El Centro 
\end{itemize}


%%%\begin{itemize}
%%%\section{Projects / internships}
%%%\item
%%%{\bf Emulation infrastructure for mobile apps \hfill  Feb 2011 to May 2012} \\
%%%\textit{Guide: Dr. Li-Shiuan Peh} \\
%%%For my Masters thesis (\url {http://web.mit.edu/anirudh/www/sm-thesis.pdf}), I attempted to build a software system to help researchers evaluate collaborative mobile apps, at scale, more repeatably. A major problem with collaborative mobile apps today is that evaluations are limited to small deployments. By replacing physical deployments with software emulation, we hope to scale evaluations to a larger number of virtual phones. This infrastructure will be capable of evaluating both cloud-based and P2P mobile applications.
%%%\item 
%%%{\bf Research Intern,} \textbf{Microsoft Reseach India} \hfill  \textbf{Summer 2010} \\
%%%\textbf{Measuring Energy consumption of Desktops in a corporate environment} \\
%%%\textit{Guides: Dr. Krishna Chintalapudi and Dr. Ranjita Bhagwan} \\
%%%I worked on Kelvin, an umbrella project that explores novel ways of profiling energy usage in a building. In an extension of my work from the previous summer, I focused on measuring Desktop energy usage in a corporate setting by building linear power consumption models for Desktops based on CPU and disk utilization. 
%%%\item 
%%%{\bf Research Intern,} \textbf{Microsoft Research India} \hfill  \textbf{Summer 2009} \\
%%%\textbf{Non-intrusive profiling of energy consumption in buildings} \\
%%%\textit{Guide: Dr. Krishna Chintalapudi } \\
%%% Fine grained profiling of energy usage in buildings can enable a host of energy efficient applications. In this project, I explored the use of sensor networks to enable fine grained energy profiling in a non intrusive manner. Specifically, I investigated the use of sensor motes based on the MSP430 processor, for estimating the power consumption of the HVAC system from temperature measurements.
%\item 
%{\bf BTech Project,} \textbf{IIT Madras} \hfill \textbf{August 2009-May 2010} \\
%\textbf{Delay Tolerant Network Routing in Open Terrains} \\
%\textit{Guide: Dr. C Siva Ram Murthy }\\
%Conventional models of routing in a delay tolerant network (DTN) assume a closed terrain where the number of nodes in a given simulation area is invariant with time. I studied the implications that an open terrain, where there is continuous churn, has on routing in DTNs. This work was published in ICDCN 2011. 
%
%\item 
%{\bf Software Engineering Project,} \textbf{IIT Madras} \hfill  \textbf{January-May 2009 }\\
% I worked on a project, under Prof D Janakiram of IIT Madras, to design and implement a middleware for wireless sensor networks as a group of four as part of my Software Engineering project. The middleware was to be service oriented and supported the basic operations of find, bind and execute. A report of the same is available at: \\
%http://www.cse.iitm.ac.in/\textasciitilde skanirud/middleware.tgz
%
%
%\item 
%{\bf Independent Research,} \textbf{IIT Madras} \hfill  \textbf{Summer 2008} \\
%\textbf{Robocup competition} \\
%We attempted to be the first Indian team to participate in Robocup , an annual International Robotics Competition. Specifically, we were trying to participate in the small size league of RoboSoccer. The report is available at: \\
% \url{http://web.mit.edu/anirudh/www/Proposal.pdf}

%\item 
%{\bf Research Intern,} \textbf{Nanyang Technological University} \hfill \textbf{Summer 2007} \\
%\textbf{Efficient segmentation of noisy iris images} \\
%\textit{Guide : Dr. Sudha Natarajan }\\
% I came up with an efficient implementation of the Iris segmentation algorithm proposed by John Daugman\footnote{High confidence visual recognition of persons by a test of statistical independence, JG Daugman, Pattern Analysis and Machine Intelligence Volume 15 Issue 11, November 1993} for comparison against new algorithms for noisy images being developed at the Center for Information Security at NTU.  This work was published in the Elsevier Signal, Image and Video Processing journal. The source code for my implementation is available at: \\ \url{http://www.mathworks.fr/matlabcentral/fileexchange/15652}
%\end{itemize}

%\section{Scholastic achievements}
%
%\begin{itemize}
% \item \textbf{Indian National Physics Olympiad} : Was awarded a Gold Medal in the Indian National Physics Olympiad(INPhO), 2006, awarded to the top 25 students(short listed from around 25000 students through two rounds).Selected for the Orientation cum Selection camp for the same held at Homi Bhabha Centre for Science Education, Mumbai.
%
%\item \textbf{Indian National Chemistry Olympiad} Was awarded a Gold Medal in the Indian National Chemistry Olympiad(INChO), 2006, awarded to the top 25 students(short listed from around 25000 students through two rounds).Selected for the Orientation cum Selection camp for the same held at Homi Bhabha Centre for Science Education, Mumbai.
%
% \item Awarded the Institute notional prize by IIT Madras for being among the top 7\% of students admitted to IIT Madras on the basis of the Joint Entrance Exam(JEE) rank.
%
% \item Secured a State rank of 4 in the state of Tamil Nadu in the All India Engineering Entrance Exam(AIEEE) 2006.
%
% \item Selected for and attended CSIR (Centre for Scientific and Industrial Research) Programme on Youth for Leadership in Science –Nov 2004 on basis of performance in CBSE Std X.
%%\end{itemize}

%\begin{itemize}
%\section{Teaching Experience}
%\item 
%{\bf Graduate Instructor, MIT EECS} \hfill \textbf{January 2012} \\
%{\bf 6.S092:  Introduction to Software Engineering in Java }
%\item 
%{\bf Teaching Assistant, MIT EECS} \hfill \textbf{Spring 2012} \\
%{\bf 6.02: Digital Communication Systems }\\
%{\bf Frederick C. Hennie III Teaching Award  }
%\end{itemize}


%\begin{itemize} \itemsep -2pt %reduce space between items
%\section{Leadership   Activities} 
%\item 
% \textbf{Core Team Member} of the organizing committee of Shaastra 2009. Shaastra is the annual technical festival of IIT Madras. In a team of three, my responsibilities included recruiting, managing and tracking the progress of 125 coordinators organizing 46 technical events in total. 
%\item
% \textbf{Coordinator} for the Robotics event at Shaastra 2008. In a team of four, my role here was to design the problem statements and ensure that the event, involving more than 100 participating teams, was smoothly conducted. 
%\end{itemize}
%
%\begin{itemize} \itemsep -2pt %reduce space between items
%\section{Co-curricular activities}
%
%\item 
% \textbf{Member} of the IIT Madras team for Robocon 2008. Robocon is India's most prestigious annual robotics competition.
%\item
% \textbf{Student Mentor} of the IIT Madras team that placed \textbf{first} in Robocon 2009.
%%\item
%% \textbf{Volunteer} for Tamil Teaching activities as part of National Service Scheme in IIT Madras.
%\end{itemize}



% Tabulate Computer Skills; p{3in} defines paragraph 3 inches wide


\end{resume} 


\end{document} 