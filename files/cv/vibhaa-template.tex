% LaTeX file for resume
% This file uses the resume document class (res.cls)
\documentclass[margin]{res} 

\usepackage{hyperref}
% the margin option causes section titles to appear to the left of body text 
\textwidth=5.0in % increase textwidth to get smaller right margin
%\usepackage{helvetica} % uses helvetica postscript font (download helvetica.sty)
%\usepackage{newcent}   % uses new century schoolbook postscript font 

\begin{document} 

\name{ Vibhaalakshmi Sivaraman\\ \hline} % the \\[12pt] adds a blank line after name 

\begin{resume} 
\hspace*{-\hoffset} 2371, Frist Campus Center \hfill  vibhaa@princeton.edu \\
\hspace*{-\hoffset} Princeton, NJ 08544\\ %\hfill 201-565-6698 \\ 

%\section{Objective} Research internship in the areas of mobile computing/networking. 
 
%\section{Objective} 
%Graduate studies (PhD) at the University of Illinois at Urbana-Champaign in the field of Computer Science in the area of networked systems.

\begin{itemize}
\section{Education} 

\item
\textbf{Princeton University, Princeton, NJ\\}
B.S.E in Computer Science (June 2017) \\
%Major GPA \textbf{3.97/4.0}, Overall GPA \textbf{3.81/4.0}\\
Coursework includes: Operating Systems, Computer Networks, Distributed Systems, Functional Programming, Theory of Algorithms, Advanced Programming Techniques, Bitcoin and Cryptocurrency Technologies
\end{itemize}

\begin{itemize}
\section{Publications}
%% Can communicated papers be put here ?
\item
\textbf{Smoking Out the Heavy-Hitter Flows with HashPipe} \\
Vibhaalakshmi Sivaraman, Srinivas Narayana, Ori Rottenstreich,  S.Muthukrishnan and Jennifer Rexford.\\
ACM Symposium on SDN Research 2017; Presented poster at the New England Networking and Systems Day 2016 \\ %\url{http://systems.cs.brown.edu/nens/2016/program/}\\
\url{https://arxiv.org/pdf/1611.04825v1.pdf}

\end{itemize}
\begin{itemize} \itemsep -2pt  % reduce space between items
\section{Projects}
\item 
{\bf Automated Testing for Forwarding Policies \hfill  Sep 2016  - present} \\
\emph{Advisor: David Walker}\\
Creating a framework that generates and runs exhaustive tests on various failure scenarios for Propane, a system that converts network-level rules on traffic forwarding policies into per-router BGP configurations.

\item
{\bf HashPipe, Heavy Hitter Detection Algorithm  \hfill Sep 2015 - Nov 2016} \\
\emph{Advisor: Jennifer Rexford}\\
Designed and prototyped an algorithm to detect heavy hitter flows on a single link on programmable data planes; evaluated the algorithms for accuracy, efficiency and feasibility on emerging programmable switches like RMT.\\
\url{https://github.com/vibhaa/iw15-heavyhitters}

\item
{\bf PassesForPasses \hfill Mar 2015 - May 2015 }\\
Created a location-based mobile and web app that allows Princeton Students to secure passes for entry into the university eating clubs. \\
\url{https://github.com/vibhaa/P4P}

 \item
{\bf BitcoinScript IDE \hfill Nov 2014 - Jan 2016 }\\
Created an IDE for Bitcoin Script that supports step-by-step execution, stack visualization, debugging and conversion to and from assembly \\ \url{https://github.com/vibhaa/bitcoinIDE}   
\end{itemize}

\begin{itemize}
\section{Industry \\Internships}
%\item
%{\bf Princeton University  - {\em{Summer Researcher}} \hfill 2016}\\
%Worked on a heavy hitter detection algorithm in Jennifer Rexford's group with ProjectX funding from the School of Engineering and Applied Sciences

\item
{\bf Microsoft (Apps and Services Team) - {\em{Summer Intern}}  \hfill  2015}\\
Added features and functionality for tables in Excel's Windows app with more touch-friendly User Interfaces than the Desktop version; implemented UI to append total rows to Excel tables, extended the UI to change table styles; extensively tested features and pushed it to production code

\item
{\bf AT\&T Labs - {\em{Summer Intern}}  \hfill 2014}\\
Designed and implemented a standalone fraud-blocking module with a REST interface that looks up call parameters in a fraud database before placing calls; designed to be independently compatible with all AT\&T application servers that process calls

\end{itemize}


\begin{itemize}
\section{Awards  and Honors}
\item
{\bf {Tau Beta Pi:}} Among the top 20\% of the senior engineering class invited to join Tau Peta Pi's Princeton Chapter {\bf \hfill Oct 2016}

\item
{\bf {SIGCOMM Student Travel Grant:}} Awarded a travel grant sponsored by NSF to attend SIGCOMM 2016 in Brazil {\bf \hfill Jun 2016}

\item
{\bf {Shapiro Prize for Academic Excellence:}} Among the top 4\% of the class of 2017 who were awarded prizes for exceptional academic achievement during sophomore year at Princeton University {\bf \hfill Oct 2015}

\item
{\bf {Google Code Jam to I/O for Women:}} Finished 16th nationwide (US) in Google Code Jam for Women. Selected (as one among 100 women) to attend Google I/O Conference 2015 {\bf \hfill Apr 2015}

\item
{\bf {HackPrinceton:}} Part of five-member team that won Social Entrepreneurship Award for an app that allows speed-reading of textbooks via Optical Character Recognition {\bf \hfill Mar 2014}

 \item
{\bf {Times Scholar (for Scholastic and Extra-Curricular achievements):}} One of 20 Times Scholars (from 45,000 applicants) selected at national level for a cash award of \$7500  {\bf \hfill Nov 2012}\\
\url{http://timesofindia.indiatimes.com/timesscholar.cms}

\item
{\bf {JEE Advanced:}} Secured All India Rank (AIR) of 230 amongst 1,500,000 students (top 0.02\%) at the IIT Joint Entrance Examination (IIT JEE) and admitted into IIT Madras	{\bf \hfill Jun 2013}
\end{itemize}

\begin{itemize}
\section {Co-curricular Activities}
\item
{\bf {Co-President, Princeton Women in Computer Science:}} Organize and coordinate programs directed at encouraging more women to pursue Computer Science and increasing retention through skill-building {\bf \hfill 2016 - Present}

\item
{\bf {Undergraduate Lab Teaching Assistant, Princeton University:}} Provide debugging and conceptual assistance on programming assignments for introductory Computer Science courses; assist the head Lab TA in the smooth functioning of the Lab TA program {\bf \hfill 2015 - Present}

\item
{\bf {Peer Academic Advisor, Princeton University:}} Provide academic assistance to freshmen and sophomores with choosing courses, majors and independent projects at Princeton University {\bf \hfill 2016 - Present}

\item
{\bf {McGraw Tutor, Princeton University:}} Tutor introductory math courses at the McGraw Center for Teaching and Learning {\bf  \hfill 2016 - Present}

\item
{\bf {Student Technology Consultant, Office of Information Technology:}} Worked with the University Tech Support Team resolving hardware and software issues with computers on campus {\bf \hfill 2014 - 2016}
\end{itemize}

\begin{itemize}
\section {Extracurricular Activities}
\item
{\bf {Classical Dancer and Teacher, Kalanjali Institute:}} Performed on stage individually and as part of a team in front of 500 plus audience receiving positive and encouraging reviews of performances in newspapers. Conducted summer classes for groups of about 15 students {\bf \hfill 2003 - Present}
\end{itemize}


%%%\begin{itemize}
%%%\section{Projects / internships}
%%%\item
%%%{\bf Emulation infrastructure for mobile apps \hfill  Feb 2011 to May 2012} \\
%%%\textit{Guide: Dr. Li-Shiuan Peh} \\
%%%For my Masters thesis (\url {http://web.mit.edu/anirudh/www/sm-thesis.pdf}), I attempted to build a software system to help researchers evaluate collaborative mobile apps, at scale, more repeatably. A major problem with collaborative mobile apps today is that evaluations are limited to small deployments. By replacing physical deployments with software emulation, we hope to scale evaluations to a larger number of virtual phones. This infrastructure will be capable of evaluating both cloud-based and P2P mobile applications.
%%%\item 
%%%{\bf Research Intern,} \textbf{Microsoft Reseach India} \hfill  \textbf{Summer 2010} \\
%%%\textbf{Measuring Energy consumption of Desktops in a corporate environment} \\
%%%\textit{Guides: Dr. Krishna Chintalapudi and Dr. Ranjita Bhagwan} \\
%%%I worked on Kelvin, an umbrella project that explores novel ways of profiling energy usage in a building. In an extension of my work from the previous summer, I focused on measuring Desktop energy usage in a corporate setting by building linear power consumption models for Desktops based on CPU and disk utilization. 
%%%\item 
%%%{\bf Research Intern,} \textbf{Microsoft Research India} \hfill  \textbf{Summer 2009} \\
%%%\textbf{Non-intrusive profiling of energy consumption in buildings} \\
%%%\textit{Guide: Dr. Krishna Chintalapudi } \\
%%% Fine grained profiling of energy usage in buildings can enable a host of energy efficient applications. In this project, I explored the use of sensor networks to enable fine grained energy profiling in a non intrusive manner. Specifically, I investigated the use of sensor motes based on the MSP430 processor, for estimating the power consumption of the HVAC system from temperature measurements.
%\item 
%{\bf BTech Project,} \textbf{IIT Madras} \hfill \textbf{August 2009-May 2010} \\
%\textbf{Delay Tolerant Network Routing in Open Terrains} \\
%\textit{Guide: Dr. C Siva Ram Murthy }\\
%Conventional models of routing in a delay tolerant network (DTN) assume a closed terrain where the number of nodes in a given simulation area is invariant with time. I studied the implications that an open terrain, where there is continuous churn, has on routing in DTNs. This work was published in ICDCN 2011. 
%
%\item 
%{\bf Software Engineering Project,} \textbf{IIT Madras} \hfill  \textbf{January-May 2009 }\\
% I worked on a project, under Prof D Janakiram of IIT Madras, to design and implement a middleware for wireless sensor networks as a group of four as part of my Software Engineering project. The middleware was to be service oriented and supported the basic operations of find, bind and execute. A report of the same is available at: \\
%http://www.cse.iitm.ac.in/\textasciitilde skanirud/middleware.tgz
%
%
%\item 
%{\bf Independent Research,} \textbf{IIT Madras} \hfill  \textbf{Summer 2008} \\
%\textbf{Robocup competition} \\
%We attempted to be the first Indian team to participate in Robocup , an annual International Robotics Competition. Specifically, we were trying to participate in the small size league of RoboSoccer. The report is available at: \\
% \url{http://web.mit.edu/anirudh/www/Proposal.pdf}

%\item 
%{\bf Research Intern,} \textbf{Nanyang Technological University} \hfill \textbf{Summer 2007} \\
%\textbf{Efficient segmentation of noisy iris images} \\
%\textit{Guide : Dr. Sudha Natarajan }\\
% I came up with an efficient implementation of the Iris segmentation algorithm proposed by John Daugman\footnote{High confidence visual recognition of persons by a test of statistical independence, JG Daugman, Pattern Analysis and Machine Intelligence Volume 15 Issue 11, November 1993} for comparison against new algorithms for noisy images being developed at the Center for Information Security at NTU.  This work was published in the Elsevier Signal, Image and Video Processing journal. The source code for my implementation is available at: \\ \url{http://www.mathworks.fr/matlabcentral/fileexchange/15652}
%\end{itemize}

%\section{Scholastic achievements}
%
%\begin{itemize}
% \item \textbf{Indian National Physics Olympiad} : Was awarded a Gold Medal in the Indian National Physics Olympiad(INPhO), 2006, awarded to the top 25 students(short listed from around 25000 students through two rounds).Selected for the Orientation cum Selection camp for the same held at Homi Bhabha Centre for Science Education, Mumbai.
%
%\item \textbf{Indian National Chemistry Olympiad} Was awarded a Gold Medal in the Indian National Chemistry Olympiad(INChO), 2006, awarded to the top 25 students(short listed from around 25000 students through two rounds).Selected for the Orientation cum Selection camp for the same held at Homi Bhabha Centre for Science Education, Mumbai.
%
% \item Awarded the Institute notional prize by IIT Madras for being among the top 7\% of students admitted to IIT Madras on the basis of the Joint Entrance Exam(JEE) rank.
%
% \item Secured a State rank of 4 in the state of Tamil Nadu in the All India Engineering Entrance Exam(AIEEE) 2006.
%
% \item Selected for and attended CSIR (Centre for Scientific and Industrial Research) Programme on Youth for Leadership in Science –Nov 2004 on basis of performance in CBSE Std X.
%%\end{itemize}

%\begin{itemize}
%\section{Teaching Experience}
%\item 
%{\bf Graduate Instructor, MIT EECS} \hfill \textbf{January 2012} \\
%{\bf 6.S092:  Introduction to Software Engineering in Java }
%\item 
%{\bf Teaching Assistant, MIT EECS} \hfill \textbf{Spring 2012} \\
%{\bf 6.02: Digital Communication Systems }\\
%{\bf Frederick C. Hennie III Teaching Award  }
%\end{itemize}


%\begin{itemize} \itemsep -2pt %reduce space between items
%\section{Leadership   Activities} 
%\item 
% \textbf{Core Team Member} of the organizing committee of Shaastra 2009. Shaastra is the annual technical festival of IIT Madras. In a team of three, my responsibilities included recruiting, managing and tracking the progress of 125 coordinators organizing 46 technical events in total. 
%\item
% \textbf{Coordinator} for the Robotics event at Shaastra 2008. In a team of four, my role here was to design the problem statements and ensure that the event, involving more than 100 participating teams, was smoothly conducted. 
%\end{itemize}
%
%\begin{itemize} \itemsep -2pt %reduce space between items
%\section{Co-curricular activities}
%
%\item 
% \textbf{Member} of the IIT Madras team for Robocon 2008. Robocon is India's most prestigious annual robotics competition.
%\item
% \textbf{Student Mentor} of the IIT Madras team that placed \textbf{first} in Robocon 2009.
%%\item
%% \textbf{Volunteer} for Tamil Teaching activities as part of National Service Scheme in IIT Madras.
%\end{itemize}



% Tabulate Computer Skills; p{3in} defines paragraph 3 inches wide


\end{resume} 


\end{document} 